\chapter[APRESENTAÇÃO DA TRAJETÓRIA DO ALUNO]{APRESENTAÇÃO DA TRAJETÓRIA DO ALUNO}

Em 2018, iniciei minha carreira na área de \ac{TI} através do curso de Engenharia de Software na \ac{PUCRS}. Porém, esse não foi exatamente o início da minha carreira profissional, mas sim uma transição. Minha primeira formação foi no curso de Bacharelado em Engenharia Mecânica pela \ac{UFRGS}, onde tive meu primeiro contato com programação de computadores. Ao longo da graduação, realizei trabalhos acadêmicos que integravam conceitos de Mecânica com programação, o que me permitiu enxergar o potencial transformador da área de TI. Em 2017, após concluir a graduação em Engenharia Mecânica, decidi seguir minha verdadeira vocação: tornar-me engenheiro de software. Como primeiro passo, ingressei no curso de Engenharia de Software na PUCRS.

Já no primeiro semestre da graduação, percebi que havia tomado a decisão certa. O ambiente acadêmico, o contato com professores e colegas e a possibilidade de atuar como bolsista de Iniciação Científica no grupo \ac{PET} e no laboratório \ac{GPIN}, além da experiência como \ac{AGES} I no projeto Simulação de Desastres, foram determinantes para consolidar essa escolha. Entre 2018 e 2019, no \ac{GPIN}, tive contato intenso com a área de Deep Learning, o que despertou meu interesse em continuar na pesquisa acadêmica. Como já possuía uma graduação, decidi candidatar-me a uma vaga de Mestrado em Ciência da Computação na \ac{PUCRS}. Fui aprovado tanto no curso quanto para atuar como bolsista no laboratório \ac{LIS}. Esse período foi marcado por pesquisas em Reconstrução 3D com modelos de Deep Learning, resultando na publicação de um artigo no \ac{IJCNN}, além de trabalhos relacionados a aplicações de realidade \ac{VR} no contexto de Aprendizado Profundo.

Ao concluir o Mestrado, iniciei minha carreira profissional como Engenheiro de Software na \ac{HP} Inc., consolidando minha transição de carreira. Após alguns meses de experiência prática no mercado, retomei o curso de Engenharia de Software em 2022, conciliando as atividades acadêmicas com minha atuação profissional. Nesse período, eu já acumulava experiência em Aprendizado de Máquina e Computação Gráfica, mas percebia a importância de consolidar conhecimentos mais amplos e aprofundados em Engenharia de Software.

Como Engenheiro de Software, pude me desenvolver em áreas como desenvolvimento web (Back-End e Front-End) utilizando tecnologias .NET, conteinerização de aplicações, segurança da informação, processos de \ac{CI/CD} e práticas de gerenciamento de projetos. Ao mesmo tempo, como \ac{AGES} II no projeto \ac{FICAI 4.0}, em 2022, tive contato com tecnologias como Angular e Spring Boot, além de aprimorar noções de \ac{UX}/\ac{UI}. Nas disciplinas da graduação, consolidei aprendizados em Testes de Software, Processamento Paralelo, Gerenciamento de Configuração e Gerenciamento de Projetos de Software.

O ano de 2023 foi marcado pelo aprofundamento em diferentes tecnologias, tanto em projetos profissionais quanto acadêmicos, incluindo um maior envolvimento com Cloud Computing. Já em 2024, participei como \ac{AGES} III no projeto Polymathech, atuando como arquiteto de software. Nessa função, pude auxiliar na definição da estrutura do projeto utilizando NodeJS, NestJS e ReactJS, além de configurar a aplicação na \ac{AWS}. Também contribuí para a organização das arquiteturas de Front-End, baseadas em componentização, e de Back-End, fundamentadas na Clean Architecture e \ac{DDD}.

Ao longo de 2025, concluí o último semestre do curso de Engenharia de Software na \ac{PUCRS} enquanto seguia atuando como Engenheiro de Software na HP Inc. Esse período marcou a convergência entre prática profissional e formação acadêmica, permitindo aplicar no trabalho conceitos avançados de arquitetura, desenvolvimento e \ac{CI/CD}, ao mesmo tempo em que aprofundava fundamentos e refletia sobre decisões de engenharia na universidade.

Minha participação na \ac{AGES} IV, no projeto Ai Produtor, consolidou essa maturidade. Nesse papel, pude exercer liderança técnica, apoiar squads, estruturar processos e conduzir decisões arquiteturais envolvendo contêineres e serviços em \ac{AWS}. A experiência reforçou a importância de organização, comunicação clara e orientação ao time, encerrando minha trajetória na AGES de forma integrada à realidade do mercado e alinhada à minha atuação profissional.
    