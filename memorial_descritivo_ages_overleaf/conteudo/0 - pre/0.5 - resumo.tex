\begin{resumo}[\protect\bfseries Resumo]
O presente documento tem como objetivo descrever o percurso do aluno Felipe Roque Tasoniero durante sua participação na disciplina AGES, destacando experiências acadêmicas, responsabilidades assumidas e aprendizados conquistados. A AGES integra o curso de Bacharelado em Engenharia de Software da PUCRS e é estruturada em quatro níveis, cada um correspondendo a funções distintas em um projeto de desenvolvimento de software. Essa proposta pedagógica proporciona um ambiente de colaboração entre alunos de diferentes semestres, que trabalham em conjunto com o objetivo comum de planejar, desenvolver e entregar uma solução tecnológica.

A dinâmica da disciplina permite que os estudantes experimentem papéis progressivos ao longo da graduação: no nível AGES I, o foco recai sobre tarefas de programação; no AGES II, na análise e administração de banco de dados; no AGES III, na definição e implementação da arquitetura de software; e no AGES IV, na gestão de equipes e de projetos. Esse formato possibilita a vivência prática de diferentes etapas do ciclo de desenvolvimento de software, promovendo não apenas o aprimoramento técnico, mas também o desenvolvimento de competências interpessoais, como comunicação, liderança, organização e colaboração.
  \bigbreak\
  \\\textbf{PALAVRAS CHAVES:}
  AGES, Engenharia de Software, Projeto, Experiência, Aprendizado, Colaboração, Gestão.
\end{resumo}