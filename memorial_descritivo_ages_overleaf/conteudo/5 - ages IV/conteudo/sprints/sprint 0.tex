\subsection{Sprint 0}

Na Sprint 0, minhas atividades como \ac{AGES} IV estiveram voltadas principalmente para dar suporte à equipe e garantir que o projeto tivesse uma base organizacional sólida. Entre as ações previstas estavam o apoio aos \ac{AGES} I, II e III no início das atividades, a organização das ferramentas de gestão do time, a estruturação inicial do backlog e o alinhamento de requisitos com os stakeholders. Além disso, suportei a documentação da Wiki com informações do projeto, organizei horários para encontros da equipe e auxiliei no suporte técnico e de gestão para a configuração dos repositórios.

Na prática, consegui organizar os alunos no AirTable \cite{airtable} e criar um template Kanban no Linear \cite{linear}, o que trouxe mais clareza para a gestão do fluxo de tarefas. Também preparei e divulguei uma tabela de horários para facilitar a marcação de encontros extraclasses, além de mapear perguntas para os stakeholders e conduzir o primeiro alinhamento com eles. Outro ponto importante foi o suporte dado aos \ac{AGES} III na criação e configuração dos repositórios no GitLab, bem como a organização inicial das páginas da Wiki. Participei ainda da criação dos fluxos de telas no Figma e organizei a divisão dos membros em quatro grupos para modelagem dos mockups. Por fim, ofereci apoio direto aos \ac{AGES} I no alinhamento de expectativas e na orientação para a reunião com os clientes.

Alguns problemas marcaram essa etapa inicial. A comunicação entre os diferentes níveis de \ac{AGES} foi difícil de alinhar no começo, e a falta de experiência de alguns membros com ferramentas como GitLab e Figma exigiu um suporte maior da minha parte. Além disso, houve desafios em conciliar os horários de todos os participantes para os encontros extraclasses e uma incerteza inicial quanto às expectativas dos stakeholders antes da primeira reunião.

Apesar dessas dificuldades, as lições aprendidas foram significativas. Percebi a importância de estruturar e organizar as ferramentas de gestão já na Sprint 0, de modo a garantir fluidez para as próximas etapas. Também ficou claro o valor do alinhamento entre os diferentes níveis de \ac{AGES}, tanto no aspecto técnico quanto no organizacional, e a necessidade de preparar previamente perguntas e tópicos para reuniões com stakeholders, garantindo objetividade e clareza. Compreendi ainda que o suporte técnico e motivacional aos membros é essencial para manter o ritmo e o engajamento do time.

Como próximos passos, ficou estabelecido que seria necessário estruturar o backlog inicial com base nas informações obtidas após a apresentação dos mockups aos stakeholders, finalizar a organização da Wiki com as primeiras definições do projeto, apoiar os \ac{AGES} III na evolução da arquitetura do código e da infraestrutura, além de elaborar os primeiros artefatos de gestão, como a \ac{EAP}/Release Plan, o Plano de Comunicação e o Plano de Respostas a Riscos.