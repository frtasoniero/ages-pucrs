\subsection{Sprint 3}

A Sprint 3 foi marcada por um avanço técnico importante, especialmente na consolidação da arquitetura de deploy e na integração dos serviços da \ac{AWS} \cite{aws}. Meu papel como \ac{AGES} IV envolveu liderar o acompanhamento dessas atividades, promover alinhamentos com os demais níveis de \ac{AGES} e oferecer suporte direto aos \ac{AGES} III e II em momentos de bloqueio técnico.

Entre as ações realizadas, destaco a criação e atualização de \textit{User Stories} relacionadas à configuração dos serviços de nuvem, a manutenção do diagrama de arquitetura de serviços da AWS e o suporte na integração do \textit{SonarCloud} com os repositórios do projeto no GitHub \cite{github}. Também participei da revisão de \textit{Merge Requests} nos módulos Back-End e Front-End, garantindo que as implementações seguissem os padrões definidos pelo time e mantivessem a consistência com a arquitetura modular proposta.

Durante essa sprint, percebi uma evolução significativa na organização interna das \textit{squads}. A separação de papéis e responsabilidades tornou a comunicação mais fluida e eficaz, permitindo que as tarefas fossem conduzidas com menor acoplamento e mais autonomia. No entanto, identifiquei oportunidades de melhoria na distribuição de tarefas: alguns membros ficaram momentaneamente sem atividades, o que atrasou parte do cronograma até o realinhamento das demandas.

Também foi notável o crescente uso de modelos de \ac{IA} para a geração de trechos de código. Embora essa prática tenha proporcionado agilidade em algumas implementações, observou-se que o uso excessivo dessas ferramentas resultou em trechos de código mais poluídos e, por vezes, inconsistentes com os padrões estabelecidos pelo projeto. Acredito que o uso de modelos de \ac{IA} deveria ter sido conduzido com cautela e análise crítica, avaliando não apenas o funcionamento imediato do código gerado, mas também sua qualidade estrutural, legibilidade e aderência às boas práticas de engenharia de software.

Além disso, percebi a necessidade de aprimorar a organização no processo de abertura de \textit{Merge Requests}. Em diversos casos, os títulos não representavam adequadamente o conteúdo das modificações realizadas, e algumas solicitações continham alterações além do escopo definido para determinada \textit{User Story}. Essa falta de padronização impactou o processo de revisão e dificultou a rastreabilidade das alterações, evidenciando a importância de boas práticas de versionamento, padronização de commits e controle de qualidade colaborativo.

As lições aprendidas nesta sprint reforçaram a importância de um planejamento mais granular e de uma gestão de backlog dinâmica, que assegure o envolvimento contínuo de todos os membros. Além disso, a experiência de acompanhar a configuração da infraestrutura em nuvem ampliou minha visão sobre a relação entre arquitetura, deploy e escalabilidade de sistemas.

Como próximos passos, concentrei esforços no suporte às melhorias para a entrega final do sistema, priorizando ajustes de qualidade no código, revisões de arquitetura e o refinamento da documentação técnica na Wiki, com o objetivo de entregar um produto mais estável, padronizado e tecnicamente coeso.