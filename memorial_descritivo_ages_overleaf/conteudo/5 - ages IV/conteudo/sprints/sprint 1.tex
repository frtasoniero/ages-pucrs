\subsection{Sprint 1}

Na Sprint 1, o foco foi apoiar os \ac{AGES} II e III no desenvolvimento das atividades técnicas, além de dar continuidade à organização das ferramentas de gestão do time, como o Linear \cite{linear} e a Wiki do GitLab. Outra frente de trabalho foi a criação de User Stories para o desenvolvimento das funcionalidades previstas, bem como o suporte na criação do diagrama de deploy na \ac{AWS}.

Durante a execução, atuei na criação e atualização das User Stories voltadas para as rotas de gestão de áreas de plantio no Back-End, ao mesmo tempo em que ofereci suporte às squads de Front-End para a integração com o Back-End. Colaborei também no desenvolvimento de testes unitários e participei ativamente das reuniões de alinhamento com os demais \ac{AGES} IV. Além disso, estive envolvido na correção de bugs que surgiram ao longo da sprint.

Entretanto, alguns problemas dificultaram a evolução do trabalho. Houve pequenas dificuldades na comunicação entre as squads, somadas à falta de retorno dos membros do Back-End sobre o andamento das tarefas. Também identifiquei baixa adesão da equipe a boas práticas de desenvolvimento, como a realização de commits frequentes, a clareza nas mensagens de commit e a execução de revisões de código. Soma-se a isso a dificuldade recorrente em conciliar os horários da equipe para encontros extraclasses.

As lições aprendidas nesta sprint foram importantes para a melhoria contínua do projeto. Notei a relevância de estruturar as User Stories logo no início da sprint, garantindo clareza no desenvolvimento e alinhamento entre os membros da equipe. Compreendi também a necessidade de reduzir o número de canais de comunicação para evitar ruídos, além de reforçar o valor do alinhamento constante entre os membros da squad. Outro ponto essencial foi a importância de manter atividades em backlog para os membros que terminam suas tarefas antes do previsto, evitando ociosidade e mantendo a produtividade.

Para as próximas etapas, o trabalho seguirá com suporte ao desenvolvimento do Back-End e apoio ao Front-End para integração com o Back-End. Além disso, estão previstos novos alinhamentos com os \ac{AGES} IV, a correção de bugs, o planejamento da Sprint 2, o suporte no desenvolvimento do deploy na \ac{AWS} \cite{aws} e a migração do código para o GitHub \cite{github}.