\subsection{Sprint 4}

A Sprint 4 representou a etapa final do projeto \textit{Ai Produtor!}, concentrando-se nas entregas conclusivas de funcionalidades e nos ajustes finais da infraestrutura da aplicação. No início da sprint, a maior parte das funcionalidades já havia sido concluída e validada pelos stakeholders, restando apenas a implementação da tela e dos serviços do módulo de \textit{Relatórios}, considerada uma das partes mais estratégicas do sistema por consolidar informações essenciais sobre safras, áreas, produtores e produtividade.

Para dar início a essa etapa, realizamos um alinhamento direto com os stakeholders a fim de detalhar quais análises, métricas e tipos de filtragem deveriam compor a página de relatórios. Com os requisitos devidamente especificados, decidimos dividir novamente a equipe em duas frentes: uma focada no desenvolvimento do módulo Front-End dos relatórios e outra dedicada ao Back-End, responsável pelas consultas, agregações e processamento dos dados provenientes do banco PostgreSQL \cite{postgresql}. Essa divisão permitiu paralelizar esforços e manter um fluxo de desenvolvimento consistente entre os módulos.

Como \ac{AGES} IV, atuei tanto no apoio técnico quanto na coordenação dessas frentes, supervisionando o desenvolvimento das novas rotas de consulta, revisando os serviços implementados no NestJS \cite{nestjs} e garantindo que o código permanecesse aderente às boas práticas de \textit{Clean Code} e \textit{Separation of Concerns}. Também conduzi revisões de nomenclatura, remoção de redundâncias e melhorias de validação, reforçando a qualidade das entregas antes da integração final ao sistema.

Outra responsabilidade importante nesta sprint foi a finalização da configuração dos serviços na \ac{AWS} \cite{aws}. Realizei revisões no ambiente de deploy, ajustes finos na estruturação dos containers Docker \cite{docker} e acompanhamento da automatização de parte dos processos de \textit{build} e \textit{deploy}. Esse refinamento foi essencial para garantir que a aplicação estivesse acessível, estável e em conformidade com a arquitetura planejada.

Apesar de ser uma sprint relativamente curta e com escopo reduzido, o rendimento da equipe acabou ficando abaixo do esperado. Algumas dificuldades técnicas e de organização de tarefas impactaram o ritmo do desenvolvimento, exigindo realinhamentos no backlog e redistribuição de atividades ao longo da sprint. Ainda assim, conseguimos manter a consistência das entregas e finalizar todas as funcionalidades planejadas dentro do prazo.

Um ponto que voltou a se destacar, assim como na Sprint 3, foi o uso excessivo de ferramentas de Inteligência Artificial para geração de código. Embora úteis em algumas situações, observou-se novamente a produção de trechos mais poluídos e inconsistentes, reforçando a necessidade de uso mais criterioso e acompanhado por análise crítica — especialmente em sprints finais, onde a estabilidade do sistema é prioridade. Além disso, práticas como titulação inadequada de \textit{Merge Requests} e inclusão de modificações além do escopo previsto persistiram em alguns momentos, indicando oportunidades importantes de melhoria no fluxo de versionamento.

Como lição geral desta sprint, percebi a importância de equilibrar a entrega de novas funcionalidades com a manutenção da qualidade técnica, especialmente no fechamento do ciclo de desenvolvimento. A necessidade de alinhamento constante entre squads, documentação clara e revisão cuidadosa do código mostrou-se fundamental para garantir um encerramento coeso e profissional do projeto.

Com a conclusão desta sprint, o sistema \textit{Ai Produtor} alcança um bom nível de maturidade e estabilidade. As entregas consolidadas refletem o esforço coletivo da equipe e reforçam os aprendizados técnicos, organizacionais e colaborativos construídos ao longo de todo o semestre.