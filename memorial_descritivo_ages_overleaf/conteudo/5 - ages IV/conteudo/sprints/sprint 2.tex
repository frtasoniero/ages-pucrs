\subsection{Sprint 2}

Durante a Sprint 2, minhas atividades como \ac{AGES} IV concentraram-se no suporte técnico e organizacional às squads, com foco na conclusão de débitos técnicos herdados da Sprint anterior e no avanço de funcionalidades essenciais para o sistema. As principais frentes de trabalho envolveram a criação e revisão de \textit{User Stories} no Linear \cite{linear}, o acompanhamento do desenvolvimento de rotas e correções no Back-End, e o apoio na configuração da infraestrutura em nuvem utilizando a \ac{AWS} \cite{aws}.

Entre as atividades executadas, destaco a revisão e aprovação de \textit{Merge Requests} no Back-End, principalmente relacionadas ao \ac{CRUD} de produtos, variedades e plantios, além da verificação de um bug que afetava o gerenciamento de containers e o processo de \textit{seed} do banco de dados. Também realizei ajustes na rota de atualização de plantações e dei suporte à integração final da aplicação para a apresentação da sprint, incluindo a correção de erros de \textit{build} no Front-End. Em paralelo, apoiei a criação das contas e configuração inicial de instâncias EC2 na AWS, parte essencial da preparação para o ambiente de deploy.

Apesar das entregas bem-sucedidas, alguns desafios foram notáveis. A comunicação entre os membros ainda apresentou atrasos, resultando em lentidão na resolução de tarefas e falta de feedback sobre dificuldades técnicas. Essas falhas impactaram parcialmente a agilidade da equipe e reforçaram a necessidade de práticas mais estruturadas de acompanhamento e \textit{follow-up} entre os níveis de \ac{AGES}.

Como lição, percebi que, embora o grupo já demonstrasse maturidade no fluxo de desenvolvimento, ainda havia lacunas no entendimento do ciclo de informações dentro da aplicação e no uso adequado das ferramentas de versionamento. Reforcei, portanto, a importância de comunicação assertiva, definição clara de prioridades e acompanhamento mais frequente das tarefas no Linear.

Para as próximas etapas, defini como foco a criação de novas \textit{User Stories} para o backlog, o suporte contínuo na configuração dos serviços da AWS, a atualização da Wiki do projeto e a implementação de melhorias voltadas à qualidade e padronização do código, incluindo revisões de estilo e estrutura nos módulos do Back-End e Front-End.
