\section[Conclusão]{Conclusão}

A atuação como \ac{AGES} IV no projeto \textit{Ai Produtor} representou uma das experiências mais completas e desafiadoras da minha formação, exigindo de mim um conjunto equilibrado de competências técnicas, organizacionais e de liderança. Desde a Sprint 0, ficou evidente que o papel de um \ac{AGES} IV vai muito além do suporte técnico: envolve estruturar processos, alinhar expectativas, orientar decisões e garantir que o trabalho de diferentes níveis de \ac{AGES} avance de forma coerente, organizada e colaborativa.

Ao longo das sprints, pude aplicar de maneira prática conhecimentos de gestão de backlog, modelagem de arquitetura, documentação técnica, comunicação com stakeholders, práticas de versionamento e automação de processos. Ferramentas como GitLab \cite{gitlab}, Linear \cite{linear}, Figma \cite{figma} e AWS \cite{aws} tornaram-se parte do trabalho cotidiano, permitindo que eu contribuísse tanto na orientação estratégica quanto na execução técnica — especialmente em momentos críticos, como na configuração da infraestrutura em nuvem e nas revisões de código em Back-End e Front-End.

As Sprints 2, 3 e 4 foram marcadas por desafios significativos, que reforçaram lições importantes sobre qualidade, processos e maturidade da equipe. Um dos pontos mais notáveis foi a crescente utilização de ferramentas de \ac{IA} para geração de código. Embora esse recurso tenha trazido agilidade, observei impactos negativos na qualidade do código, na legibilidade e na padronização. Esse cenário reforçou a importância do uso criterioso de \ac{IA}, acompanhado de revisão crítica e de aderência às boas práticas de engenharia de software — especialmente em projetos colaborativos, onde consistência e clareza são fundamentais para a evolução do sistema.

Outro aprendizado importante esteve relacionado ao fluxo de versionamento do projeto. Em diversas situações, identifiquei \textit{Merge Requests} com títulos inadequados, escopos desalinhados com as \textit{User Stories} e modificações excessivas dentro de um único \textit{Merge Request}. Essas falhas impactaram diretamente o processo de code review, aumentando o tempo de análise e reduzindo a rastreabilidade das alterações. Essa recorrência evidenciou a necessidade de reforçar processos formais de versionamento e revisão, bem como a importância de educar continuamente o time para boas práticas de colaboração e controle de qualidade.

No campo das soft skills, a minha atuação exigiu mediação entre níveis de \ac{AGES}, alinhamento de expectativas, condução de reuniões e apoio direto aos membros que enfrentavam dificuldades técnicas ou bloqueios. A comunicação entre squads, por vezes falha, gerou atrasos e exigiu intervenções para reorganizar tarefas e evitar gargalos. Essas experiências reforçaram a importância de uma comunicação clara, objetiva e frequente — além da necessidade de manter o backlog atualizado e distribuído de forma equilibrada para evitar períodos de ociosidade ou sobrecarga.

Também ficou evidente o quanto a organização inicial da Sprint 0 impactou positivamente as etapas posteriores. Estruturar ferramentas, definir templates, padronizar processos e criar um ambiente colaborativo desde o início foi fundamental para sustentar o desenvolvimento técnico das sprints seguintes. Da mesma forma, a vivência em disciplinas como Gerenciamento de Projetos, Arquitetura de Software e Banco de Dados ofereceu a base teórica necessária para decisões que influenciaram diretamente na qualidade do produto entregue.

Por fim, o projeto \textit{Ai Produtor} consolidou-se como uma experiência transformadora. Não apenas pelo desenvolvimento técnico do sistema — que alcançou maturidade, estabilidade e confiabilidade ao término da Sprint 4 — mas, principalmente, pela evolução profissional que pude vivenciar. Aprendi a liderar com empatia, a apoiar com responsabilidade, a tomar decisões sob pressão e a equilibrar visão técnica com necessidades humanas e organizacionais.

Concluir o Ai Produtor como AGES IV me permitiu enxergar, com clareza, a dimensão de responsabilidade envolvida em um papel de liderança e integração entre múltiplos níveis. Diferente das experiências anteriores, aqui o desafio não estava apenas em orientar tecnicamente, mas em garantir coerência organizacional, qualidade do código e alinhamento contínuo entre stakeholders e desenvolvimento. As dificuldades enfrentadas com versionamento, uso excessivo de IA, padronização de entregas e gestão de tarefas mostraram que a maturidade de um time depende tanto de orientação quanto de disciplina processual. Essa experiência ampliou minha visão sobre liderança Técnica e Estratégica, reforçando que coordenar pessoas e práticas é tão essencial quanto dominar ferramentas e tecnologias. O AGES IV marcou, assim, uma transição clara na minha formação: de desenvolvedor atuante para profissional capaz de conduzir equipes e decisões de forma crítica e estruturada.