\section[Conclusão]{Conclusão}

O projeto \textit{Simulação de Desastres}, desenvolvido na disciplina AGES I, marcou meu primeiro contato com um ambiente de desenvolvimento colaborativo real e estruturado. Essa experiência representou um ponto de virada na minha formação, pois me permitiu vivenciar de forma prática os desafios de um projeto de software conduzido por uma equipe multidisciplinar, com diferentes níveis de experiência e responsabilidades. Ao longo do semestre, compreendi que o sucesso de um projeto não depende apenas do domínio técnico, mas também da capacidade de comunicação, da clareza de papéis e da organização coletiva do grupo.

O desenvolvimento do simulador exigiu o aprendizado de novas tecnologias e paradigmas. Foi meu primeiro contato com a linguagem \textit{Python}, com o framework \textit{Flask} e com o uso de \textit{WebSockets} por meio do \textit{Flask-SocketIO}, que possibilitaram a comunicação em tempo real entre agentes e servidor. Além do aprendizado técnico, o projeto me proporcionou uma visão mais ampla sobre arquitetura de sistemas distribuídos, testes automatizados e a importância da documentação como instrumento essencial para o alinhamento e continuidade do trabalho em equipe.

Nos primeiros momentos, predominava um sentimento de incerteza quanto à complexidade do simulador e às responsabilidades individuais. Entretanto, conforme as tarefas foram sendo divididas, ficou evidente o quanto a integração entre os membros era determinante para o avanço do projeto. Mesmo diante das dificuldades, pude contribuir com a documentação técnica, a criação de testes unitários e a proposição de soluções para os problemas de comunicação entre os agentes e o simulador.

Apesar dos progressos obtidos, o projeto enfrentou entraves significativos relacionados à gestão e à comunicação da equipe. A ausência de uma liderança constante, o distanciamento de alguns integrantes mais experientes e a centralização das decisões em poucos membros acabaram impactando o ritmo e a qualidade das entregas. Além disso, a falta de  acompanhamento efetivo das tarefas dificultou a visualização do progresso geral.

Para futuras edições de projetos semelhantes, considero essencial a adoção de práticas que reforcem a colaboração e a transparência entre os membros da equipe. Seria recomendável definir papéis de liderança técnica e organizacional mais bem delimitados, garantindo uma melhor distribuição de responsabilidades e o acompanhamento das atividades. Por fim, o incentivo à cultura de feedback e à troca de conhecimento entre os níveis contribuiria para uma equipe mais engajada, coesa e preparada para lidar com desafios complexos.

Encerrar o AGES I evidenciou que meu maior aprendizado não esteve nas tecnologias, mas na percepção de como falhas de organização, liderança e comunicação comprometem um projeto. Essa experiência reforçou que a maturidade do processo e a colaboração são tão importantes quanto a técnica. Saio do AGES I consciente de que trabalhar em equipe é tão desafiador quanto indispensável para atuar de forma profissional.