\subsection{Sprint 2}

Durante a Sprint 2, foram iniciados o desenvolvimento do motor de simulação e a implementação da estrutura dos agentes. No motor da simulação foram incorporados os geradores de eventos, responsáveis por criar situações dinâmicas durante os turnos de execução. Paralelamente, minha equipe ficou encarregada de implementar as classes de agentes, definindo a estrutura necessária para a execução das ações disponíveis, como locomoção, coleta de imagens, resgate de vítimas, entre outras.

Nessa etapa, eu ainda enfrentava certa dificuldade de comunicação com alguns colegas mais experientes, principalmente no esclarecimento de dúvidas técnicas relacionadas à arquitetura do sistema e às tecnologias utilizadas. Apesar disso, recebi apoio constante dos alunos de nível AGES IV, que auxiliaram no entendimento do funcionamento geral da aplicação e na organização das tarefas. 

Como ainda havia algumas barreiras técnicas para que eu pudesse contribuir diretamente com o desenvolvimento das principais funcionalidades, assumi a responsabilidade pela implementação da estrutura de testes unitários. Juntamente com outro colega de nível AGES I, dedicamos boa parte do tempo dessa sprint à construção de testes para validar as ações realizadas pelos agentes, como movimentação e coleta de dados. Essa experiência foi fundamental para compreender a importância da verificação automática de código e do papel dos testes na garantia da qualidade e confiabilidade do sistema.
