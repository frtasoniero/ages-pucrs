\subsection{Sprint 1}

Durante a Sprint 1, os integrantes de nível AGES IV decidiram dividir a equipe em dois grupos, a fim de otimizar o processo de desenvolvimento. O primeiro grupo, do qual eu fiz parte, ficou responsável pela criação da estrutura de comunicação entre os agentes e a \ac{API}. O segundo grupo dedicou-se à implementação do núcleo do simulador, que seria o componente central do sistema. 

No início dessa sprint, a equipe ainda não possuía uma definição concreta da arquitetura do simulador nem dos mecanismos específicos de comunicação que seriam utilizados. Diante dessa incerteza, o grupo responsável pela comunicação iniciou um processo de análise e planejamento, buscando compreender quais informações precisariam ser trocadas entre os agentes e o servidor. A partir dessas discussões, começamos também a estruturar a documentação técnica do projeto na Wiki, descrevendo as decisões tomadas e os formatos de mensagens previstos para a interação entre as partes do sistema.

Durante essa etapa, enfrentei algumas dificuldades relacionadas à comunicação com os colegas dos níveis AGES II e III, especialmente quando necessitava de apoio técnico sobre as tecnologias adotadas no projeto. Essa limitação reforçou a importância de uma comunicação mais integrada entre os diferentes níveis da equipe, um aspecto essencial em projetos de desenvolvimento colaborativo. 

Paralelamente, mantive o foco nos estudos dirigidos sobre Python e Flask, aprofundando meus conhecimentos nas ferramentas que seriam fundamentais para o andamento do projeto. Também contribuí na organização e atualização da Wiki, registrando informações sobre as primeiras decisões arquiteturais e o comportamento esperado do sistema. Além disso, da mesma forma que na Sprint 0, não tivemos nenhuma entrega concreta da aplicação, concentrando nossos esforços em alinhar com o stakeholder a forma como estávamos estruturando o simulador e o modelo de comunicação. Esta sprint foi, portanto, um momento de consolidação do aprendizado técnico e de amadurecimento na dinâmica de trabalho em equipe, servindo de base para o desenvolvimento mais estruturado das etapas seguintes.

