\subsection{Sprint 4}

Na Sprint 4, a equipe decidiu interromper o desenvolvimento de novos testes para concentrar os esforços na correção de erros e na estabilização do funcionamento do simulador. O objetivo principal dessa etapa foi garantir uma comunicação consistente entre o simulador e os agentes, uma vez que ainda enfrentávamos diversos problemas relacionados à manipulação das informações transmitidas e recebidas durante a execução. Para simplificar o processo e assegurar resultados mais controlados, optamos por implementar apenas algumas ações básicas para os agentes, como locomoção e coleta de água.

Durante essa última sprint, busquei auxiliar os colegas em diferentes tarefas, especialmente na identificação e correção de falhas no código. No entanto, ainda percebi dificuldades de comunicação e um menor nível de engajamento por parte de alguns membros mais experientes, o que impactou diretamente o ritmo de desenvolvimento. 

Ao término da sprint, não conseguimos concluir integralmente as entregas propostas pelo stakeholder. O simulador ainda apresentava comportamentos inconsistentes e carecia da implementação de diversas ações adicionais para os agentes, bem como de uma transmissão completamente estável das informações entre os componentes do sistema. Apesar disso, essa etapa final foi importante para consolidar aprendizados sobre trabalho colaborativo, gestão de prioridades e resolução de problemas em ambientes de desenvolvimento de software, encerrando a experiência da AGES I com uma visão mais realista dos desafios presentes em projetos complexos.
