\subsection{Sprint 3}

Durante a Sprint 3, a equipe demonstrou um certo desânimo em relação ao andamento do projeto, o que impactou o ritmo de desenvolvimento. Nessa etapa, iniciamos a implementação do processo de comunicação entre os agentes e o simulador. Para essa finalidade, foi adotada a comunicação via \textit{WebSockets}, utilizando os recursos oferecidos pelo \textit{Flask-SocketIO}. 

A escolha por essa tecnologia trouxe diversas vantagens ao projeto. Diferentemente do modelo tradicional de requisições \textit{HTTP}, os \textit{WebSockets} permitem uma comunicação bidirecional e em tempo real entre cliente e servidor. Essa característica foi fundamental para o simulador, uma vez que os agentes precisavam enviar e receber informações continuamente a cada turno de simulação, sem a necessidade de estabelecer novas conexões a cada interação. Além disso, o uso do \textit{Flask-SocketIO} simplificou a integração com o framework já utilizado, garantindo maior desempenho e escalabilidade à aplicação.

Com o avanço do desenvolvimento, surgiram diversos erros de execução e inconsistências na troca de mensagens, o que levou a equipe a concentrar seus esforços na realização de testes e na correção dos problemas de comunicação. Ao final da sprint, constatamos que nenhuma \textit{User Story} havia sido concluída integralmente, uma vez que o foco se voltou para a estabilização do sistema. 

Apesar das dificuldades enfrentadas, percebi uma melhora significativa na minha comunicação com os colegas mais experientes, o que me permitiu contribuir de forma mais efetiva nas discussões técnicas. Pude propor ideias e possíveis soluções para os problemas de sincronização e envio de mensagens entre os agentes e o simulador, o que representou um importante avanço na minha participação e no meu desenvolvimento dentro do projeto.
