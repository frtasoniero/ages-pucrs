\subsection{Sprint 0}

A Sprint 0 marcou o primeiro contato da equipe com o projeto \textit{Simulação de Desastres}. Nessa etapa inicial, ocorreu a apresentação do projeto pelo stakeholder, que compartilhou a proposta, os objetivos gerais e as expectativas em relação ao produto final. A partir desse encontro, iniciou-se um processo de discussão coletiva para compreender as necessidades do projeto, identificar possíveis abordagens de implementação e definir tecnologias adequadas para o seu desenvolvimento. 

O stakeholder forneceu uma base sólida para o início dos trabalhos, disponibilizando fluxogramas Figura \ref{fig:proposta-fluxo-simulador}, imagens explicativas e algumas \textit{User Stories} já definidas, o que auxiliou a equipe a visualizar de maneira mais clara os requisitos e funcionalidades esperadas do simulador. A partir dessas informações, o grupo deu início ao processo de organização e estruturação do projeto, estabelecendo o ambiente de desenvolvimento e discutindo as primeiras estratégias de implementação.

Como integrante do nível AGES I, minha principal responsabilidade era atuar como desenvolvedor e auxiliar na documentação do sistema. Embora eu já tivesse tido algum contato prévio com projetos relacionados à área de Simulação Computacional, essa experiência representou meu primeiro envolvimento em um desenvolvimento colaborativo com papéis e responsabilidades formalmente definidos. Logo percebi que o sucesso do projeto dependia fortemente de uma comunicação eficaz e de uma boa integração entre os membros da equipe.

Durante essa sprint, tive também meu primeiro contato com a linguagem de programação Python e com o framework Flask, que seriam amplamente utilizados no projeto. Com o apoio de colegas mais experientes, utilizei esse período para estudar os fundamentos dessas tecnologias e compreender suas principais estruturas e padrões de uso. Além disso, busquei contribuir com o time na análise do comportamento esperado do simulador. No entanto, essa atividade revelou-se desafiadora, pois quanto mais detalhávamos o funcionamento do sistema, mais dúvidas surgiam sobre os processos internos e a interação entre seus componentes.

Foi também nesse momento que entrei em contato com diversos conceitos que até então eram novos para mim, como o funcionamento de uma \ac{API}, os protocolos de comunicação, os tipos de dados trocados entre os agentes e o servidor, e o processamento realizado pelo núcleo do simulador. Apesar das incertezas, essa sprint foi fundamental para o amadurecimento técnico e conceitual da equipe. Ao final dessa fase, ainda não havia uma estrutura completamente definida do sistema, mas a reunião de encerramento com o stakeholder foi essencial para esclarecer dúvidas e alinhar expectativas, consolidando a base para as etapas seguintes do projeto.

\begin{figure}[H]
    \centering
    \small
    \caption{Proposta enviada pelo stakeholder para o fluxo de funcionamento do simulador.}
    \includegraphics[width=1\linewidth]{conteudo//2 - ages I//conteudo//figures/proposta-fluxo-simulador.jpg}
    Fonte: Wiki do projeto.
    \label{fig:proposta-fluxo-simulador}
\end{figure}