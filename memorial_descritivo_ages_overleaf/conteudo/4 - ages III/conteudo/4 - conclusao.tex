\section[Conclusão]{Conclusão}

Ao refletir sobre minha participação neste projeto, percebo que pude contribuir de maneira significativa para a estruturação da aplicação, atuando tanto nas etapas iniciais quanto nas fases finais do desenvolvimento — o que estava diretamente alinhado às minhas responsabilidades como arquiteto de software do projeto. Também participei ativamente da construção do módulo Back-End, da configuração do processo de conteinerização da aplicação e das revisões de código realizadas pelos demais membros da equipe.

Além das contribuições técnicas, destaco o papel relevante que desempenhei na passagem de conhecimento aos colegas de equipe. Essa troca constante de informações e experiências foi essencial para o avanço coletivo e para a resolução de bloqueios técnicos durante o desenvolvimento. Nesse sentido, percebo que evoluí consideravelmente em competências interpessoais, aprimorando minha capacidade de comunicação, liderança técnica e colaboração em equipe.

Assim como em projetos anteriores, notei a aplicação direta de conhecimentos adquiridos em diversas disciplinas do curso de Engenharia de Software, como Construção de Software, Arquitetura e Desenvolvimento de Software e Fundamentos de Redes de Computadores. Somam-se a isso as experiências adquiridas em minha atuação profissional como engenheiro de software, que contribuíram para uma visão mais prática e assertiva das decisões técnicas tomadas ao longo do projeto.

Apesar de termos concluído com sucesso todas as funcionalidades propostas para o Polymathech, identifiquei um ponto relevante de melhoria relacionado à arquitetura de serviços da \ac{AWS}. Em alguns momentos, precisei oferecer suporte a outras atividades, o que reduziu o tempo que eu poderia dedicar à implementação completa dos serviços em nuvem. Assim, embora tenhamos conseguido disponibilizar a aplicação funcionalmente na \ac{AWS}, acredito que poderíamos ter aprimorado a solução ao empregar outros serviços que otimizassem o desempenho geral do sistema — por exemplo, no processamento e renderização de imagens, na gestão eficiente das informações de banco de dados e na melhoria do tempo de resposta da aplicação.

Finalizar o Polymathech consolidou minha compreensão de que assumir um papel técnico mais central — especialmente como arquiteto — envolve, inevitavelmente, equilibrar profundidade técnica com apoio às necessidades do time. Embora eu tenha conseguido garantir estabilidade na arquitetura e auxiliar na evolução do Back-End e do processo de conteinerização, ficou claro que o envolvimento em múltiplas frentes impõe limites reais à dedicação em temas mais complexos, como a arquitetura em nuvem da AWS. Essa percepção reforçou a importância de priorização e de delegação inteligente, competências essenciais para cargos de liderança técnica. No AGES III, mais do que codificar, aprendi a orientar, ouvir e tomar decisões que influenciam diretamente o fluxo de trabalho — um amadurecimento significativo em minha trajetória.