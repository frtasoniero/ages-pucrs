\section[Introdução]{Introdução}

O projeto \textit{Polymathech} surgiu da necessidade de enfrentar uma problemática que afeta milhares de jovens em todo o Brasil: a dificuldade na escolha da carreira profissional a seguir. Essa decisão, muitas vezes tomada sem orientação adequada, tem impactado diretamente os índices de permanência e conclusão no Ensino Superior.

A partir da análise de dados disponibilizados pelo \ac{INEP}, foi possível identificar que, na última década, apenas cerca de 30\% dos estudantes matriculados em Instituições de Ensino Superior concluíram sua graduação, enquanto mais da metade desistiu do curso ou migrou para outra área. Esses dados evidenciam a importância de oferecer uma orientação vocacional mais acessível e eficaz, especialmente voltada ao público jovem, para auxiliá-lo na definição de caminhos profissionais de acordo com suas aptidões e interesses.

Com base nesse contexto, o projeto \textit{Polymathech} foi idealizado com o objetivo de desenvolver uma plataforma web voltada à orientação profissional de jovens de forma atrativa, realista e interativa. A aplicação busca combinar uma linguagem clara e próxima da juventude com recursos tecnológicos que permitam a realização de testes vocacionais, o acesso a conteúdos educacionais e um ambiente de estudo dinâmico e acolhedor. Dessa forma, o sistema pretende contribuir para a redução da evasão universitária, promovendo uma escolha de carreira mais consciente e assertiva.

O projeto teve seu período de execução entre os meses de março e julho de 2024 e contou com a participação dos stakeholders Vinícius da Rosa Soares e Arthur Battistel Ilha, da professora orientadora Cristina Nunes e de uma equipe composta por 15 alunos dos cursos de Engenharia de Software e Sistemas de Informação da \ac{PUCRS}, conforme ilustrado na Figura~\ref{fig:projeto-time-polimathech}.


\begin{figure}[H]
    \centering
    \small
    \caption{Equipe de desenvolvimento e Stakeholder - Polymathech}
    \includegraphics[width=1\linewidth]{conteudo//4 - ages III//conteudo//figures/equipe-polymathech.jpg}
    Fonte: https://tools.ages.pucrs.br/polymathech/wiki/-/wikis/home
    \label{fig:projeto-time-polimathech}
\end{figure}