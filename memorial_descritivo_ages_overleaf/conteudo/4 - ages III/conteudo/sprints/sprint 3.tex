\subsection{Sprint 3}

Durante a Sprint 3, muitas das funcionalidades planejadas para o projeto já se encontravam praticamente concluídas, resultado do elevado desempenho da equipe na sprint anterior. Assim, as principais demandas concentraram-se no desenvolvimento e ajuste de novas páginas no módulo Front-End e na configuração do processo de deploy da aplicação na \ac{AWS}.

Meu principal objetivo nesta sprint foi iniciar o desenvolvimento das configurações necessárias para o deploy da aplicação. Juntamente com os membros da minha \textit{Squad}, configurei o uso de containers Docker \cite{docker} tanto para o módulo \ac{API} quanto para o módulo \ac{UI}, preparando o ambiente para execução integrada dos serviços.

Durante o desenvolvimento dessas configurações, enfrentamos alguns desafios técnicos, como a comunicação entre containers e o gerenciamento adequado das variáveis de ambiente. Além disso, dediquei parte significativa do tempo à revisão de \textit{Merge Requests} criados para os módulos Front-End e Back-End, o que acabou reduzindo o tempo disponível para o avanço das configurações de deploy.

Outro ponto relevante desta sprint foi o avanço na criação de testes unitários para as rotas do Back-End, bem como a configuração de um \textit{pipeline} no GitLab, responsável por executar o \textit{build} do módulo Back-End a cada novo \textit{Merge Request}. Essa automação contribuiu para o aumento da qualidade do código e para a detecção precoce de erros durante o desenvolvimento.

Como reflexão final, identifiquei que um dos principais pontos a melhorar nesta sprint foi a baixa delegação de tarefas. Dediquei uma parcela considerável do tempo a revisões de código que poderiam ter sido redistribuídas entre os demais membros da equipe, o que teria permitido maior foco no avanço do processo de deploy e no refinamento das configurações da aplicação.