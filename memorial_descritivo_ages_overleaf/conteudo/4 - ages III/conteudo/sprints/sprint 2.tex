\subsection{Sprint 2}

Durante a Sprint 2, após a reorganização dos alunos em suas respectivas \textit{Squads}, meu primeiro objetivo foi repassar conhecimentos técnicos aos colegas que passaram a ser responsáveis pelo desenvolvimento de novas funcionalidades no módulo Back-End da aplicação. Essa etapa inicial foi importante para alinhar práticas de desenvolvimento e garantir que todos tivessem domínio sobre as ferramentas e padrões utilizados.

Concluída essa fase, concentrei meus esforços nas atividades diretamente sob minha responsabilidade, que consistiram em: (1) desenvolver novas rotas no Back-End para a geração dos resultados dos testes vocacionais, e (2) realizar a integração dessas funcionalidades com o módulo Front-End.  

Como primeiro passo, optamos por atualizar a modelagem do banco de dados, com o intuito de otimizar as consultas necessárias para o processamento dos resultados dos testes. Assim, elaboramos um novo modelo \ac{ER} e seu correspondente modelo lógico, incluindo as entidades, atributos e relacionamentos essenciais para o funcionamento completo da aplicação. Nessa etapa, também foram atualizados os scripts de migração de dados no Prisma ORM \cite{prisma}, garantindo a criação adequada das estruturas no banco de dados PostgreSQL \cite{postgresql}.

Por fim, realizei a integração das novas rotas do Back-End com o Front-End, permitindo a geração e exibição dinâmica dos resultados dos testes vocacionais. Embora a sprint tenha resultado em entregas relevantes, observou-se que a comunicação entre alguns membros da equipe foi limitada em determinados momentos, o que ocasionou um acúmulo de tarefas próximo ao encerramento da sprint.
