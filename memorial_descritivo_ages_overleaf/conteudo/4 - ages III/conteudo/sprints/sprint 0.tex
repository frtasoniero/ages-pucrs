\subsection{Sprint 0}

Durante a primeira sprint do projeto \textit{Polymathech}, atuei no desenvolvimento de diversas atividades fundamentais para a estruturação inicial da aplicação, entre elas: estudos dirigidos, organização dos repositórios, documentação e configuração dos módulos Front-End e Back-End, definição dos processos de versionamento de código, elaboração das arquiteturas de cada módulo, apoio na criação das primeiras \ac{US} e suporte aos colegas dos níveis \ac{AGES} I e \ac{AGES} II.

Na etapa inicial, realizei estudos dirigidos com o objetivo de compreender quais modelos de arquitetura e tecnologias seriam mais adequados ao desenvolvimento do projeto. Esses estudos foram realizados por meio de vídeos, cursos rápidos e materiais disponibilizados nas disciplinas do curso de Engenharia de Software da \ac{PUCRS}. Com base nesse aprendizado, iniciei a estruturação dos módulos Front-End e Back-End, configurando seus diretórios e dependências principais.

Paralelamente, organizei os repositórios do projeto, concedendo acesso aos membros da equipe, definindo o fluxo de versionamento, papéis e responsabilidades, além de documentar os procedimentos de configuração de ambiente. Essa documentação teve como objetivo facilitar a integração dos integrantes e garantir que todos pudessem configurar e executar os módulos localmente de forma padronizada e eficiente.

Como etapa complementar da sprint, e considerando que parte das minhas atribuições como \ac{AGES} III já havia sido concluída, pude contribuir também no desenvolvimento dos protótipos da aplicação. Atuei na padronização de componentes e dimensões de tela, além de incluir fluxos automatizados de navegação entre as páginas no \textit{Figma} \cite{figma}. Essa funcionalidade foi importante para proporcionar aos stakeholders e aos demais membros da equipe uma visão mais realista e interativa da aplicação em sua versão final.