\subsection{Sprint 1}

Durante a Sprint 1, apresentei ao time a estrutura inicial dos módulos Front-End e Back-End do projeto, em uma reunião remota que teve como objetivo alinhar o entendimento sobre a arquitetura proposta e esclarecer dúvidas técnicas. Esse momento foi essencial para que a equipe pudesse propor melhorias e identificar oportunidades de otimização no desenvolvimento da aplicação.

Com a divisão da equipe em \textit{Squads} — pequenos grupos responsáveis por funcionalidades específicas — passei a atuar na implementação das rotas de cadastro de usuário no módulo Back-End. Nessa etapa, contribuí na criação das rotas da \ac{API}, na configuração da conexão com o banco de dados PostgreSQL \cite{postgresql}, na execução das migrações e na aplicação de mecanismos de validação e segurança dos dados, como a utilização de senhas criptografadas.

O rápido progresso desse módulo permitiu validar as entregas junto aos alunos \ac{AGES} IV. Após a validação, a equipe foi redistribuída para apoiar outras \textit{Squads}. Nessa nova etapa, atuei novamente no Back-End, desta vez no desenvolvimento das funcionalidades de login e autorização de usuários. Implementamos o sistema de autenticação com tokens \ac{JWT}, que garante maior segurança ao permitir a troca de informações criptografadas entre os módulos Front-End e Back-End, utilizando chaves públicas e privadas para validação. Além disso, foi implementada a verificação de permissões de acesso com base no papel do usuário, restringindo determinadas rotas conforme o tipo de perfil (usuário comum ou administrador).

Com a conclusão dessas atividades, realizamos uma nova rodada de validações junto aos \ac{AGES} IV. A partir daí, passei a apoiar o desenvolvimento no módulo Front-End, colaborando na correção de falhas e na integração entre os módulos. Com os dois ambientes devidamente integrados, executamos testes de ponta a ponta (\textit{End-to-End}) para identificar pendências e garantir o correto funcionamento das funcionalidades implementadas.
